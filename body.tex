\section{Introduction}

Charge-coupled devices convert incoming photons into electrons that drift under the influence of an electric field toward potential wells located beneath the collecting electrodes. In an ideal detector, each pixel would act as an isolated charge bucket whose effective area remains constant as the signal increases. In practice, this picture breaks down: as charges accumulate in a pixel, they modify the local electrostatics and slightly repel subsequent drifting electrons. The resulting deflection of drift lines shifts the pixel boundaries and redistributes charge among neighboring pixels. This phenomenon, known as the Brighter–Fatter effect \citep{Antilogus2014, Guyonnet2015}, causes bright sources to appear systematically broader and introduces measurable correlations between neighboring pixel signals in uniformly illuminated images.

The effect was first recognized through two observational signatures. The first is a departure from Poisson statistics in flat fields, where the variance of pixel values grows more slowly than the mean, revealing a transfer of variance into positive pixel–pixel covariances \citep{Downing2006}. The second is a flux-dependent increase in the apparent width of point-like sources, where bright stars or laboratory spots appear larger than faint ones \citep{Antilogus2014, Lupton2014, Astier2013}. These two manifestations are different projections of the same physical process: the local distortion of pixel boundaries by stored charge.

Although the change in apparent size is small, it has profound consequences for high-precision astronomical measurements. Weak gravitational lensing analyses, which aim to measure galaxy shapes to sub-percent accuracy, are directly sensitive to flux-dependent point-spread function distortions \citep{Mandelbaum2015}. A fractional change of one percent in the PSF width translates to a comparable systematic error in measured ellipticities if not properly modeled \citep{Jarvis2014, Mandelbaum2018}. For surveys such as the Rubin Observatory Legacy Survey of Space and Time, the cumulative impact of this subtle effect could exceed the systematics budget unless corrected at the pixel level \citep{Betoule2014}.

The underlying mechanism of the Brighter–Fatter effect is now understood as a consequence of electrostatic charge redistribution in the fully depleted silicon bulk of modern CCDs \citep{Rasmussen2016, Lage2017}. The strength and spatial structure of these distortions depend on both the geometry of the collecting wells and the electric field configuration inside the sensor. Because the field lines also depend on where the charges are generated, the effect can vary with the wavelength of the incident light \citep{Stubbs2014}. Shorter wavelengths, which convert near the front surface, experience longer drift paths and stronger lateral deflection; longer wavelengths penetrate deeper, where the drift field is more vertical and the induced deflection smaller. Quantifying this wavelength dependence is essential to ensure that brighter–fatter corrections derived from one illumination condition (for example, a red LED flat field) remain valid across the full optical range used in science observations.

Several empirical and electrostatic models have been proposed to describe the redistribution of charge and to reconstruct the effective pixel boundary shifts \citep{Guyonnet2015, Gruen2015, Coulton2018, Astier2019, Lage2021}. The current generation of corrections, implemented in astronomical pipelines, generally assumes that the effect is achromatic and that boundary displacements scale linearly with stored charge. These assumptions are increasingly inadequate as measurements push toward sub-percent precision. A physically motivated, wavelength-dependent model is required to reproduce the behavior of real sensors and to guarantee consistent corrections across filters.

In this work, we use commissioning data from the LSST Camera to investigate the wavelength dependence of the Brighter–Fatter coefficients. By combining flat-field covariance measurements with an electrostatic pixel-boundary model, we aim to quantify how the illumination wavelength modifies the effective distortion strength. This study extends previous analyses by incorporating the measured photon-conversion depth distributions of the LSST LEDs into the modeling of drift-field perturbations. The resulting framework provides a unified description of the Brighter–Fatter effect valid across the wavelength range relevant for LSST science operations.


\vskip 0.4in



This is the Rubin Observatory overview paper: \citet{2019ApJ...873..111I}.
