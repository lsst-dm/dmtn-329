
\begin{abstract}
The brighter-fatter effect (BF) is a subtle yet critical manifestation of pixel-to-pixel correlations in CCDs, where brighter sources induce a redistribution of charge that broadens object shapes. Caused by electrostatic deflections from accumulated charge, the BF effect introduces flux-dependent distortions that can bias precision astronomical measurements, notably in weak lensing surveys. In this study, we investigate the influence of the incident wavelength on BF coefficients, using both LSST camera data and electrostatic modeling. We explore how variations in illumination modify the local electric field within the CCD, thereby altering charge drift paths. We are developing an electrostatic model to quantify these distortions. Our results show that BF coefficients depend not only on charge accumulation but also on the incident wavelength, adding a new layer of complexity to their interpretation. This has direct implications for calibration strategies and the accuracy of shape measurements in current and future imaging surveys.
\end{abstract}

